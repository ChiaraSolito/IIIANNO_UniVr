\documentclass{article}

\usepackage[T1]{fontenc}
\usepackage[utf8]{inputenc}
\usepackage{graphicx}
\usepackage{booktabs, siunitx}
\usepackage[shortlabels]{enumitem}
\usepackage[margin=0.5in]{geometry}
\usepackage{amsmath}

\begin{document}
    \begin{center}
        \vspace{0.5cm}
        \LARGE
        \textbf{Segnali e Immagini}
        Domande in stile esame\\
        \vspace{0.5cm}
        \small
        \textbf{Chiara Solito - Bioinformatica A.A. 2021/22}
    \end{center}


\section*{Domande stile tema d'esame - Teoria}

\subsection*{Esercizio 1}
Fornire le definizioni della tassonomia dei segnali.
\subsection*{Svolgimento}
Classificazione dei segnali con definizione:
\begin{itemize}
    \item Un segnale a tempo continuo assume ogni valore dell’asse $x$, quindi non c’è nessuna pausa e nessun limite; un
    esempio può essere l’onda sonora, quando acquisisco questo tipo di segnale, devo campionare tutto l’intervallo
    temporale che prendo in considerazione, un intervallo di tempo può essere visto come un certo insieme di istanti,
    quantità finita. Lo rappresento in scrittura come
\end{itemize}
\end{document}