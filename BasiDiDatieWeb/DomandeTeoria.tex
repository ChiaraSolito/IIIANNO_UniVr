\documentclass{article}

\usepackage[T1]{fontenc}
\usepackage[utf8]{inputenc}
\usepackage{graphicx}
\usepackage{booktabs, siunitx}
\usepackage[shortlabels]{enumitem}
\usepackage[margin=0.5in]{geometry}
\usepackage{amsmath}

\begin{document}
    \begin{center}
        \vspace{0.5cm}
        \LARGE
        \textbf{Basi di Dati e Web\\}
        Domande in stile esame\\
        \vspace{0.5cm}
        \small
        \textbf{Chiara Solito - Bioinformatica A.A. 2021/22}
    \end{center}


\section*{Domanda n.1}
Illustrare lo scopo e le funzionalità del Gestore dell'Affidabilità in un sistema di gestione di basi di dati.
\subsection*{Risposta}

\section*{Domanda n.2}
Illustrare le caratteristiche del Log.
\subsection*{Risposta}

\section*{Domanda n.3}
Illustrare le principali anomalie delle transazioni concorrenti.
\subsection*{Risposta}

\section*{Domanda n.4}
Illustrare le funzionalità del Gestore del Buffer e le primitive da esso supportate.
\subsection*{Risposta}

\section*{Domanda n.5}
Si riporti lo schema dei componenti di un DataBase Management System (DBMS) coinvolti nella gestione delle
interrogazioni e nell'accesso alla memoria secondaria.\\
Si commenti opportunamente lo schema, descrivendo brevemente ogni componente.
\subsection*{Risposta}

\section*{Domanda n.6}
Si illustrino brevemente le strutture ad albero e si descrivano indici primari e secondari.
\subsection*{Risposta}

\section*{Domanda n.7}
Nel contesto della gestione delle transazioni, si descrivano i possibili guasti e i meccanismi per la loro gestione.
\subsection*{Risposta}

\end{document}