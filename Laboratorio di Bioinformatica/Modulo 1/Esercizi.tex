\documentclass{article}

\usepackage[T1]{fontenc}
\usepackage[utf8]{inputenc}
\usepackage{graphicx}
\usepackage{booktabs, siunitx}

\begin{document}
\section{Traccia}
In seguito ad una variazione locale di pH la proteina $P_1$ suisce un cambiamento conformazionale che la rende attiva. La forma attiva della proteina ($P_{1a}$)
è in grado di legare la proteina $P_2$ in modo reversibile con le costanti cinetiche $k_as = 3.4 \times 10^5 M^{-1}s^{-1}$ e $k_dis = 5 \times 10^{-2}s^{-1}$.\\
Sapendo che le concentrazioni iniziali della protina $P_1$ e della proteina $P_2$ sono rispettivamente $35 \mu M$ e $24 \mu M$ e che la cost6ante cinetica di variazione conformazionale della proteina $P_1$ è $kconf = 7 \times 10^3 M^{-1}s^{-1}$
simulare il sistema dinamicamente e rispondere ai seguenti quesiti:
\begin{enumerate}
   \item Sono sufficienti 0.2 secondi (200 ms) per stabilire l'equilibrio?
   \item Assumendo che a t=20s il sistema sia all'equilibrio, determinare empiricamente la costante di equilibrio e confrontarla con la costante di equilibrio teorica.
   \item Supponendo ora che la proteina $P_{1a}$ sia soggetta a degradazione con una costante cinetica $kdeg=1.2 \times 10^{-1} M^{-1}s^{-1}$, per quanto tempo nel sistema si può rilevare la presenza di $P_3$?
\end{enumerate}
\section{Soluzione}
\begin{verbatim}
*****************MODEL NAME

*****************MODEL NOTES

*****************MODEL STATE INFORMATION
$P1(0)=25e-6   %M (moli/I)$
$P2(0)=24e-6   %M (moli/I)$

*****************MODEL PARAMETERS
k_conf=7e3 \%M^-1 s^-1
k_as=3.4e5 \%M^-1 s^-1
k_dis=5e-2 \%s^-1
k_deg=1.2e-2 %aggiunto successivamente
                 %all'esercizio 3

*****************MODEL VARIABLES

*****************MODEL REACTIONS
P1 => P1a : r1
   vf=k_conf*P1
P1a + P2 <=> P3: r2
   vf=k_as*P1a*P2
   vr=k_dis*P3
P1a => :r3     %aggiunto successivamente
   vf=k_deg*P1a    %all'esercizio 3

*****************MODEL FUNCTIONS

*****************MODEL EVENTS

*****************MODEL MATLAB FUNCTIONS
\end{verbatim}
\section{Risposte}
\paragraph{Domanda 1}
Sono sufficienti per la prima reazione sono sufficienti, ma per la seconda no. Né $P_3$ né $P_{1a}$ hanno raggiunto l'equilibrio. Già dopo 10 secondi invece è visibile l'equilibrio raggiunto da tutte le specie.
Quindi non sono sufficienti per stabilire l'equilibrio dell'intero sistema.
\paragraph{Domanda 2}
$$k_{eq} = \frac{ P_3}{P_{1a}*P_2} = 6.8 \times 10^6$$
$$K_{as}*P_{1a}*P_2 = k_{dis}*P_3$$

Dobbiamo ora calcolare la costante di equilibrio empirica:\\
$$P_3(20) c.a. = 2.3x10^{-5}$$
$$P_{1a}(20) c.a. = 1.13 \times 10^{-5}$$
$$P_2(20) c.a. = 3.08 \times 10^{-7}$$
$$k_{eq} = \frac{ P_3(20)}{P_{1a}(20)*P_2(20)} = \frac{2.3 \times 10^{-5}}{1.13 \times 10^{-5}*3.08 \times 10^{-7}} == 6.8 \times 10^{6}$$

\paragraph{Domanda 3}
Per capirlo aggiungiamo la reazione 3, con una nuova costante $k_{deg}$.\\
Dopo circa $10^5$ secondi (quindi circa 28 ore) abbiamo raggiunto lo zero (più o meno) per $P_3$.\\
Spiegazione: dopo la prima reazione, di dissociazione di $P_1$, $P_{1a}$ tende a dissiparsi, quindi non è possibile dopo le 28 ore che si formi $P_3$ e quindi poi tende a sparire.
\end{document}