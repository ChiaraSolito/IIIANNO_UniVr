\documentclass{article}

\usepackage[T1]{fontenc}
\usepackage[utf8]{inputenc}
\usepackage{graphicx}
\usepackage{booktabs, siunitx}
\usepackage[margin=0.90in]{geometry}
\usepackage{etoolbox,titling}
\usepackage{enumitem}
\usepackage{fancyhdr}
\usepackage{soulutf8}
\usepackage{tcolorbox}
\usepackage{xcolor}
\tcbuselibrary{skins}

\definecolor{boxTitle4}{HTML}{d7d7c1}
\definecolor{boxBackground4}{HTML}{ebebe0}
\definecolor{boxFrame4}{HTML}{c3c3a2}

\tcbset{box4/.style={
    enhanced, fonttitle=\bfseries,
    colback=boxBackground4, colframe=boxFrame4,
    coltitle=black, colbacktitle=boxTitle4,
    attach boxed title to top left={xshift=0.3cm,
                                    yshift*=-\tcboxedtitleheight/2},
    boxed title style={
      before upper=\hspace*{1.5cm}, % reserve space for the image
      overlay={
       \node at ([xshift=0.1cm]frame.west)
         {\includegraphics[scale=0.65]{bc-loupe}};
      }
    }
  }
}

\newtcolorbox{box4}[1][]{box4, #1}



\begin{document}
\begin{center}
    \vspace*{1cm}
    \LARGE
    \textbf{Presentazione}

\end{center}
\Large
\section*{Intro: Cos'è Open Targets Genetics?}
Open Target Genetics è {l’ultima release della piattaforma Open Targets}: una partnership tra pubblico e privato
che utilizza i dati genetici e genomici umani per l’identificazione sistematica e la prioritizzazione dei bersagli
farmacologici.\\
Ma come lo fa questa piattaforma e perché in maniera così innovativa da risultare in una menzione in NAR?\\
La maggior parte delle varianti, individuate attraverso i GWAS, si trova nella parte non codificante del
genoma: ciò suggerisce che tali varianti vadano ad intaccare tratti complessi, alterando l’espressione
dei geni vicini, attraverso meccanismi di regolazione, e influenzando in maniera significativa le malattie
studiate dai GWAS. Identificare un gene causale è difficile poiché bisogna integrare dati dai GWAS
con dati di trascrittomica, proteomica ed epigenomica prendendo in considerazione un’ampia tipologia
cellulare o tissutale.In assenza di un portale già esistente che consenta di rispondere sistematicamente a
un’ampia gamma di domande biologiche, è stato costruito OTG sulla base della tecnologia più recente
per consentire di aggiungere e sfogliare facilmente i dati.
\section*{Come funziona?}
Le pagine del database si dividono in 3 pagine di ricerca, che vediamo qui: Studio, Lead e Tag Variants e Gene. Queste si collegano a 2 pagine interne molto importanti, che vedremo poi come sono fatte.
\normalsize
\begin{box4}
    [title={\textbf{Ricerca per Studio}}]
    {Iniziamo la ricerca a partire dallo studio per:
    \begin{itemize}
        \item Visualizzare i loci associati a un tratto nello studio selezionato
        \item Identificare i geni prioritari implicati funzionalmente da ciascun locus
        \item Visualizzare il 95\% di set credibili (se disponibili) e proxy in ogni locus
    \end{itemize}}
\end{box4}
\begin{box4}
    [title={\textbf{Ricerca per Variante}}]
    {Iniziamo la ricerca a partire dalla variante per:
    \begin{itemize}
        \item Identificare un elenco classificato di geni funzionalmente implicati dalla variante
        \item Visualizzare e analizzare i dati funzionali mediante i quali i geni sono assegnati a questa variante
        \item Visualizzare i risultati PheWAS per la variante nella biobanca britannica
        \item Visualizzare la struttura del collegamento intorno alla variante
    \end{itemize}}
\end{box4}
\begin{box4}
    [title={\textbf{Ricerca per Gene}}]
    {Iniziamo la ricerca a partire dal gene per:
    \begin{itemize}
        \item Identificare i loci che implicano funzionalmente un gene
        \item Collegarti a informazioni dettagliate sul gene e sui farmaci che lo prendono di mira
        \item Identificare in quali tratti questo gene può svolgere un ruolo, in base alle varianti a cui è assegnato
    \end{itemize}}
\end{box4}

\newpage
\begin{center}
    {\color{red} \LARGE Parte di Aurelia sulle Pipeline}
\end{center}
\newpage

\begin{center}
    \vspace*{1cm}
    \LARGE
    \textbf{Video}

\end{center}
\Large
\section*{Study}
Cercando tramite lo studio, abbiamo per prima cosa il \textbf{Summary} con info generali (come l'ID e la grandezza dello studio).\\
La parte più importante però è il \textbf{Manhattan Plot}, che rappresenta i \textbf{\textit{loci associati indipendentemente}} che superano il livello di significatività (la linea rossa) dei GWAS: l'asse delle x sono i \textbf{cromosomi}.\\
Sotto troviamo la \textbf{tabella riassuntiva}, coni dettagli completi dei locus, in cui ogni riga è una \textbf{variant lead}.
\section*{Compare Studies}
Identifichiamo rapidamente i \textbf{loci sovrapposti} (possiamo selezionare anche più studi da confrontare).\\
I loci condivisi sono segnati in rosso e dettagliati nella tabella sottostante.
\section*{Variant}
La ricerca per variante si può fare tramite il suo \textbf{Locus} oppure tramite il suo \textbf{ID Ensembl (nome)}.\\
La prima tabella \textbf{Assigned Genes}, mi mostra quali sono i geni \underline{funzionalmente implicati} da questa variante.\\
Sotto abbiamo un \textbf{plot dei PheWAS associati}, in cui ogni triangolo rappresenta l'associazione della variante a un tratto, con uno studio relativo.\\
Infine ho due tabelle molto importanti:
\begin{itemize}
    \item La prima mi dice \textbf{quali lead variant hanno questa variante come Tag}
    \item La seconda mi dice \textbf{quali varianti tag sono associate a questa variante come lead}
\end{itemize}
\section*{Gene}
Sopra abbiamo sempre le informazioni, a cui subito sotto si aggiungono tutti i \textbf{collegamenti al resto della piattaforma Open Target}.\\
La prima tabella degli Studi Associati al gene è \textbf{ordinata secondo lo score assegnato dalla pipeline L2G}, mentre sotto l'altra tabella ci mostra quelli \textbf{associati tramite la pipeline di colocalizzazione}.\\
Entrambe associano quindi lo studio al fenotipo.
\newpage
\begin{center}
    {\color{red} \LARGE Parte di Aurelia sul video}
\end{center}
\newpage
\end{document}