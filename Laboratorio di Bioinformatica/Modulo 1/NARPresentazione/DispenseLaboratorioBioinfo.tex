\documentclass{article}

\usepackage[T1]{fontenc}
\usepackage[utf8]{inputenc}
\usepackage{graphicx}
\usepackage{booktabs, siunitx}
\usepackage{tikz}
\usepackage{tikz-qtree}
\usepackage{pifont}
\usepackage[margin=0.90in]{geometry}
\usepackage{etoolbox,titling}
\usepackage{enumitem}
\usepackage{fancyhdr}
\usepackage{soulutf8}

\pagestyle{fancy}
\fancyhf{}
\rhead{Chiara Solito}
\lhead{Dispense di Laboratorio di Bioinformatica}
\rfoot{Pagina \thepage}
\lfoot{Bioinformatica - A.A. 2021/22}
\usetikzlibrary{trees}
\tikzstyle{every node}=[draw=black,thick,anchor=west]


\begin{document}
\newcommand\tab[1][0.3cm]{\hspace*{#1}}


\begin{titlepage}
    \begin{center}
        \vspace*{1cm}
            
        \Huge
        \textbf{Laboratorio di Bioinformatica}
            
        \vspace{0.5cm}
        \LARGE
        Dispense del corso
            
        \vspace{1.5cm}
            
        \textbf{Chiara Solito}

        \vspace{0.8cm}

            
        \Large
        Corso di Laurea in Bioinformatica\\
        Università degli studi di Verona\\
        A.A. 2021/22
            
    \end{center}
\end{titlepage}
La presente è una dispensa riguardante il corso di \textbf{Laboratorio di Bioinformatica} del CdS in Bioinformatica (Università degli Studi di Verona). Per la stesura di questa dispensa si è fatta fede al materiale didattico fornito direttamente dal professore nell'Anno Accademico 2021/2022. Eventuali variazioni al programma successive al suddetto anno non saranno quindi incluse.\\
Insieme a questo documento in formato PDF viene fornito anche il codice \LaTeX  con cui è stato generato.
\tableofcontents
\thispagestyle{empty}
\newpage
\thispagestyle{empty}
\section{Il corso}
Il corso si propone di presentare allo studente le basi teoriche e applicative di algoritmi e programmi utilizzati nella ricerca e nell’analisi dei dati contenuti nelle principali banche dati biologiche di uso cor-rente. Il corso si compone di due moduli di seguito specificati.\\
Modulo 1: In questo modulo verranno appresi gli strumenti volti all’utilizzo dell’informazione in prote-omica, genomica, biochimica, biologia molecolare e strutturale. Si fornisce inoltre un’introduzione all’analisi e la visualizzazione di dati strutturali relativi a macromolecole biologiche e loro complessi e la creazione di semplici modelli dinamici e statici di reti biomolecolari, che avvicinerà lo studente all’emergente disciplina della systems biology.\\
Modulo 2: In questo modulo lo studente acquisirà conoscenza pratica degli strumenti bioinformatici per l'analisi, l'interpretazione e la predizione di dati biologici in proteomica, genomica, biochimica, biologia molecolare e strutturale. 
In particolare, gli studenti avranno la possibilità di applicare stru-menti della boinformatica allo stato dell'arte a specifici problemi biologici.
\begin{titlepage}
    \begin{center}
        \vspace*{1cm}
        \LARGE
        \textbf{Lezione 1: Introduzione}
            
        \vspace{1.5cm}
        
        \large
        Ripasso delle basi e introduzione dei concetti fondamentali

        \vspace{0.8cm}

    \end{center}
\end{titlepage}
\section{Cos'è la bioinformatica?}
La bioinformatica è (oggi) una disciplina scientifica dedicata alla risoluzione di problemi biologici a livello
molecolare con metodi informatici. Descrive fenomeni biologici in modo numerico/statistico.
\\
La bioinformatica principalmente:
    \begin{itemize}
        \item Fornisce modelli per l'interpretazione di dati provenienti da esperimenti di biologia molecolare e biochimica al fine di identificare tendenze e leggi numeriche
        \item genera nuovi strumenti matematici per l'analisi di sequenze di DNA, RNA e proteine (frequenza di sequenze rilevanti, loro evoluzione e funzione).
        \item organizza le conoscenze acquisite in basi di dati al fine di rendere tali dati accessibili a tutti, ottimizzando gli algoritmi di ricerca dei dati
    \end{itemize}
Condivide alcuni argomenti con:
    \begin{itemize} 
        \item \textbf{Systems biology}
            \subitem Rappresenta i processi biologici come sistemi per comprenderne le funzioni e i principi in modo olistico per mezzo di modelli matematici
        \item \textbf{Computational biology}
            \subitem Integra i risultati sperimentali con quelli derivanti da esperimenti in silico, ottenuti quindi per mezzo di metodi informatici a partire da dati biologici.
    \end{itemize}

\begin{titlepage}
    \begin{center}
        \vspace*{1cm}
        \LARGE
        \textbf{Lezione 6: Allineamenti Multipli di Sequenze}

    \end{center}
\end{titlepage}
\section{Allineamento multiplo di sequenze}
\subsection{Visione Generale}
\subsubsection{Una definizione}
Un allineamento multiplo è una collezione di tre o più sequenze proteiche (o nucleotidiche) parzialmente o completamente allineate
\begin{itemize}
    \item I residui e le zone omologhe sono allineate in colonne per tutta la lunghezza delle sequenze
    \item Il senso dell’omologia dei residui è evoluzionistico
    \item Il senso dell’omologia dei residui è strutturale
\end{itemize}
Si tratta di un argomento di ricerca attivo dagli anni '90.
\subsubsection{Alcuni fatti}
Non c'è necessariamente un allineamento "corretto" per una famiglia di proteine.\\
\textbf{Perchè?}
    \begin{itemize}
        \item Le sequenze di proteine evolvono
        \item Le corrispondenti strutture tridimensionali evolvono, anche se più lentamente
        \item Può essere particolarmente difficile identificare i residui che si sovrappongono nello spazio (strutturalmente) in un allineamento multiplo di sequenze.
    \end{itemize}
\hl{Due proteine che condividono il 30\% di identità di sequenza avranno circa il 50\% dei residui sovrapponibili nelle due strutture}
\subsubsection{Caratteristiche utili per realizzzarlo}
Alcuni residui allineati, come cisteine che formano ponti disolfuro, o i triptofani, possono essere altamente conservati
    \begin{itemize}
        \item Ci possono essere motivi conservati come un dominio transmembrana
        \item Alcune caratteristiche come le strutture secondarie, siti attivi e di legame per ligandi o complessi sono spesso conservate
        \item Ci possono essere regioni con inserimenti o delezioni propagati in parte della famiglia.
        \item I principi che vedremo sono focalizzati sulle proteine ma sono validi in generale anche per sequenze nucleotidiche.
    \end{itemize}
\subsubsection{Utilizzi e Vantaggi}
    \begin{itemize}
        \item Il MSA è più sensibile di quello a coppie nel rilevamento di omologie, per questo è uno strumento essenziale nella costruzione di modelli strutturali per omologia
        \item L’output di BLAST può assumere la forma di un MSA, e possono essere individuati residui conservati o motivi
        \item In un MSA si possono analizzare i dati di una popolazione
        \item Una singola query può essere cercata contro un database di MSA (ad esempio Pfam)
        \item Le regioni regolatorie dei geni sono spesso identificabili da MSA
    \end{itemize}
\subsection{Metodi}
I metodi esatti non vengono trattati in questa sede: non ci sono soluzioni efficienti e già con 5 sequenze il tempo di computazione è eccessivo (esponenziale)
\subsubsection{Metodi Euristici}
\hl{\textbf{Metodi progressivi}}: usano un albero guida (analogo ad un albero filogenetico) per determinare come combinare uno per uno allineamenti a coppie
(progressivamente) per creare un allineamento multiplo.\\
\small{Esempi: CLUSTAL OMEGA (W), MUSCLE (usato da HomoloGene)}
\paragraph{Il MSA progressivo di Feng-Doolittle (1987) alla base di Clustal (W) avviene in 3 fasi}
\begin{enumerate}
    \item Realizzare una serie di allineamenti a coppie globali (Needleman e Wunsch, algoritmo di programmazione dinamica) di cui si calcola la distanza (matrice delle distanze)
    \item Creare un albero guida a partire dalla matrice delle distanze
    \item Allineare progressivamente le sequenze
\end{enumerate}
\textbf{MSA progressivo, fase 1 di 3:}\\
\normalsize{generare allineamenti a coppie globali}\\
\small{Esempio: allineare 5 globine (1, 2, 3, 4, 5).}\\
\paragraph{\hl{Primo step:} a due a due e valutare gli score di ogni possibile allineamento a coppie\\}
\textbf{Numero di allineamenti a coppie necessari per coprire tutte le possibili combinazioni}
\begin{itemize}
    \item Per n sequenze, (n-1) (n) / 2
    \item Per 5 sequenze, (4) (5) / 2 = 10
    \item Per 200 sequenze, (199) (200) / 2 = 19.900
\end{itemize}…Quindi per molte sequenze ClustalW è molto lento ed è preferibile usare metodi più veloci (MUSCLE è molto veloce).
\paragraph{\hl{Secondo step:} albero guida\\}
\textbf{Convertire i punteggi di similitudine in punteggi di distanza:} è matematicamente più semplice, oltre che più intuitivo, lavorare con le distanze. Una semplice definizione di distanza è data dalla percentuale di
residui diversi (100- SI in \%) che viene inserita nella matrice delle distanze.

\end{document}