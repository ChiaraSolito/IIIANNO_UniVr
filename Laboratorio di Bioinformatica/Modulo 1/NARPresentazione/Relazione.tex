\documentclass{article}

\usepackage[T1]{fontenc}
\usepackage[utf8]{inputenc}
\usepackage{graphicx}
\usepackage{booktabs, siunitx}
\usepackage{tikz}
\usepackage{tikz-qtree}
\usepackage{pifont}
\usepackage[margin=0.90in]{geometry}
\usepackage{etoolbox,titling}
\usepackage{enumitem}
\usepackage{fancyhdr}
\usepackage{soulutf8}

\pagestyle{fancy}
\fancyhf{}
\rhead{Chiara Solito}
\lhead{Dispense di Laboratorio di Bioinformatica}
\rfoot{Pagina \thepage}
\lfoot{Bioinformatica - A.A. 2021/22}
\usetikzlibrary{trees}
\tikzstyle{every node}=[draw=black,thick,anchor=west]


\begin{document}
\newcommand\tab[1][0.3cm]{\hspace*{#1}}


\begin{titlepage}
    \begin{center}
        \vspace*{1cm}
            
        \Huge
        \textbf{Laboratorio di Bioinformatica - Presentazione NAR}
            
        \vspace{0.5cm}
        \LARGE
        Relazione associata
            
        \vspace{1.5cm}
            
        \textbf{Chiara Solito e Aurelia Timis}

        \vspace{0.8cm}

            
        \Large
        Corso di Laurea in Bioinformatica\\
        Università degli studi di Verona\\
        A.A. 2021/22
            
    \end{center}
\end{titlepage}
La presente è una relazione riguardante la presentazione nell'ambito dei database trattati nell'issue di. Per il corso di \textbf{Laboratorio di Bioinformatica} del CdS in Bioinformatica (Università degli Studi di Verona). Per la stesura di questa dispensa si è fatta fede al materiale didattico fornito direttamente dal professore nell'Anno Accademico 2021/2022. Eventuali variazioni al programma successive al suddetto anno non saranno quindi incluse.\\
Insieme a questo documento in formato PDF viene fornito anche il codice \LaTeX  con cui è stato generato.
\tableofcontents
\thispagestyle{empty}
\newpage
\thispagestyle{empty}
\section{Introduzione}
La maggior parte delle varianti, individuate attraverso i GWAS, si trova nella parte non codificante del genoma: ciò suggerisce che tali varianti vadano ad intaccare tratti complessi, alterando l'espressione dei geni vicini, attraverso meccanismi di regolazione, e influenzando in maniera significativa le malattie studiate dai GWAS. 
Identificare un gene causale è difficile poiché bisogna integrare dati dai GWAS con dati di trascrittomica, proteomica ed epigenomica prendendo in considerazione un'ampia tipologia cellulare o tissutale.In assenza di un portale già esistente che consenta di rispondere sistematicamente a un'ampia gamma di domande biologiche, è stato costruito OTG sulla base della tecnologia più recente per consentire di aggiungere e sfogliare facilmente i dati. 
\end{document}