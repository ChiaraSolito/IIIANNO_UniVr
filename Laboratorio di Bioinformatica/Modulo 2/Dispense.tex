\documentclass{article}

\usepackage[T1]{fontenc}
\usepackage[utf8]{inputenc}
\usepackage{graphicx}
\usepackage[font=small,labelfont=bf]{caption}
\usepackage{booktabs, siunitx}
\usepackage{tikz}
\usepackage{tikz-qtree}
\usepackage{pifont}
\usepackage[margin=0.90in]{geometry}
\usepackage{etoolbox,titling}
\usepackage{enumitem}
\usepackage{fancyhdr}
\usepackage{soulutf8}
\usepackage{epigraph}
\usepackage{amssymb}
\usepackage{amsmath}

\pagestyle{fancy}
\fancyhf{}
\rhead{Chiara Solito}
\lhead{Dispense di Laboratorio di Bioinformatica}
\rfoot{Pagina \thepage}
\lfoot{Bioinformatica - A.A. 2021/22}
\usetikzlibrary{trees}
\tikzstyle{every node}=[draw=black,thick,anchor=west]
\newcommand{\angstrom}{\mbox{\normalfont\AA}}

\begin{document}
\newcommand\tab[1][0.3cm]{\hspace*{#1}}


\begin{titlepage}
    \begin{center}
        \vspace*{1cm}
            
        \Huge
        \textbf{Laboratorio di Bioinformatica}
            
        \vspace{0.5cm}
        \LARGE
        Dispense del corso - Modulo 2
            
        \vspace{1.5cm}
            
        \textbf{Chiara Solito}

        \vspace{0.8cm}

            
        \Large
        Corso di Laurea in Bioinformatica\\
        Università degli studi di Verona\\
        A.A. 2021/22
            
    \end{center}
\end{titlepage}
\section{Introduzione}
La Bioinformatica è una scienza che risolve problemi biologici attraverso l’elaborazione di materiale proveniente da esseri viventi. Questo materiale è in genere sono sequenze di DNA. L’utilizzo dell’informatica è necessario per elaborare la grande quantità dei dati.
Per questo è bene ricordare che l’informazione di partenza sono le sequenze proteiche o nucleotidiche. L’obbiettivo ultimo è quello di estrarre informazioni dalle sequenze. Tali sequenze rappresentano l’informazione disponibile e possono essere:
\begin{itemize}
    \item Sequenze genomiche
    \item Sequenze proteiche
    \item Immagini
    \item Informazioni provenienti da System Biology
    \item Informazioni di carattere evoluzionistico
\end{itemize}
In questo modulo ci concentreremo sull’annotazione genomica e sulla bioinformatica strutturale e poi anche il risequenziamento del genoma.


\end{document}